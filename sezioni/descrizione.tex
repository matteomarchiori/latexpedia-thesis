\chapter{Nomecapitolo\label{sec:labelcapitolo}}
\lettrine[findent=1.5em]{T}esto.

    \section{Nomesezione\label{sec:labelsezione}}
    
    Esempio di figura:
    \begin{figure}[htp]
        \centering
        \includegraphics[width=\textwidth]{img/nomefigura.pdf}
        \caption[Caption figura per menu]{Caption figura mostrata contestualmente}
        \label{fig:labelfigura}
    \end{figure}
    
    Paragrafo
    
    Paragrafo
    
    \subsection{Nomesottosezione\label{sec:labelsottosezione}}
    
    Esempio lista:

    \begin{itemize}
        \item item:
            \begin{enumerate}
                \item subitem.
                \label{lst:labelsublist}
            \end{enumerate}
        \label{lst:labellist}
    \end{itemize}
    
    Esempio counter:
    
    \newcounter{i}
    \setcounter{i}{1}
    
    \subsection{Subsection\label{sec:labelsubsection}}
    \begin{itemize}
        \item RO$0\thei$: incrementa contatore \stepcounter{i}
        \item RO$0\thei$: incrementa contatore \stepcounter{i}
        \label{lst:labellist}
    \end{itemize}
    
    Riporto il contatore a 1:
    \setcounter{i}{1}
    
    \subsection{Subsection \label{sec:labelsubsection}}
    \begin{itemize}
        \item RD$0\thei$: incremento il contatore \stepcounter{i}
        \label{lst:labellist}
    \end{itemize}
    
    Testo evidenziato: \textit{backend}

    Esempio figura sulla destra:
    
    \begin{wrapfigure}[5]{r}{2cm}
        \centering
        \def\svgwidth{1.5cm}
        \input{img/immagine_generata_con_inkscape}
        \caption[Caption menu per immagine]{\unskip}
        
        \begin{small}
            Caption contestuale
        \end{small}
        \label{fig:labelimmagine}
    \end{wrapfigure}
    Testo.
    
    \bigskip
    Testo
    \bigskip\bigskip
    
    Da usare per andare a pagina nuova:
    \pagebreak

    Per nuova pagina "stiracchiando" il contenuto:
    \clearpage

    Esempio acronimo: \ac{DBMS}