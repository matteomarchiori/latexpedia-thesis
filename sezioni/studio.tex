Esempio di citazione:
Una definizione di architettura a microservizi è la seguente:

    \begin{quoting}
        \omissis insieme di piccoli servizi, ognuno dei quali esegue in un processo separato, che comunicano attraverso messaggi (ad esempio chiamate HTTP attraverso un'\ac{API}). Tali servizi vengono progettati attorno alle capacità di business e sono rilasciabili in modo indipendente \omissis \parencite{labelbibliografia}
    \end{quoting}

Citazione da bibliografia: \parencite{fowler:articoloMicroservizi}.

Esempio immagine:

\begin{figure}[htp]
    \centering
    \includegraphics[width=\textwidth]{img/monolite.pdf}
    \caption[Architettura monolitica]{Architettura monolitica}
    \label{fig:monolite}
\end{figure}

Esempio figura multipla:

\begin{figure}[htp]
    \centering
    \subfloat[Testo][Testo]
    {\includegraphics[width=.45\textwidth]{img/orchestra}} \quad
    \subfloat[Testo][Testo]
    {\includegraphics[width=.45\textwidth]{img/ballet}} \quad
    \caption{Caption}
    \label{fig:label}
\end{figure}
\pagebreak

Esempio figura multipla:

\begin{figure}[htp]
    \centering
    \subfloat[Testo][Testo]
    {\includegraphics[width=.45\textwidth]{img/nomeimmagine}} \quad
    \subfloat[Testo][Testo]
    {\includegraphics[width=.45\textwidth]{img/nomeimmagine}} \quad
    \subfloat[Testo][Testo]
    {\includegraphics[width=\textwidth]{img/nomeimmagine}} \quad
    \caption{Caption}
    \label{fig:label}
\end{figure}