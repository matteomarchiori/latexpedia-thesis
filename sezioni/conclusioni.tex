\chapter{Nomecapitolo\label{sec:labelcapitolo}}
\lettrine[findent=1.5em]{T}esto.

\section{Nomesezione\label{sec:labelsezione}}

Questo è un numero in cifre: $304$.
Questo è un acronimo: \ac{CFU}.

\begin{table}[htp]
    \centering
    \normalsize
    \begin{tabularx}{\textwidth}{Xl}
        \toprule
        \textbf{Titolo} & \textbf{Titolo}\\
        \midrule
        contenuto & \textcolor{green}{contenuto}\\
        \midrule
        contenuto & \textcolor{green}{contenuto}\\
        \bottomrule
    \end{tabularx}
    \caption{Caption}
    \label{tab:labeltabella}
\end{table}

Esempio di riferimento a sezione (si veda la sezione~\vref{sec:labelsezione}).

Con due a capo il testo rientra.
Con un a capo invece si rimane nello stesso paragrafo.

Esempio di lista:
\begin{itemize}
    \item item;
    \item item.
    \label{lst:labellista}
\end{itemize}